% Options for packages loaded elsewhere
\PassOptionsToPackage{unicode}{hyperref}
\PassOptionsToPackage{hyphens}{url}
\PassOptionsToPackage{dvipsnames,svgnames,x11names}{xcolor}
%
\documentclass[
]{article}

\usepackage{amsmath,amssymb}
\usepackage{iftex}
\ifPDFTeX
  \usepackage[T1]{fontenc}
  \usepackage[utf8]{inputenc}
  \usepackage{textcomp} % provide euro and other symbols
\else % if luatex or xetex
  \usepackage{unicode-math}
  \defaultfontfeatures{Scale=MatchLowercase}
  \defaultfontfeatures[\rmfamily]{Ligatures=TeX,Scale=1}
\fi
\usepackage{lmodern}
\ifPDFTeX\else  
    % xetex/luatex font selection
\fi
% Use upquote if available, for straight quotes in verbatim environments
\IfFileExists{upquote.sty}{\usepackage{upquote}}{}
\IfFileExists{microtype.sty}{% use microtype if available
  \usepackage[]{microtype}
  \UseMicrotypeSet[protrusion]{basicmath} % disable protrusion for tt fonts
}{}
\makeatletter
\@ifundefined{KOMAClassName}{% if non-KOMA class
  \IfFileExists{parskip.sty}{%
    \usepackage{parskip}
  }{% else
    \setlength{\parindent}{0pt}
    \setlength{\parskip}{6pt plus 2pt minus 1pt}}
}{% if KOMA class
  \KOMAoptions{parskip=half}}
\makeatother
\usepackage{xcolor}
\usepackage[top=30mm,left=20mm,heightrounded]{geometry}
\setlength{\emergencystretch}{3em} % prevent overfull lines
\setcounter{secnumdepth}{5}
% Make \paragraph and \subparagraph free-standing
\ifx\paragraph\undefined\else
  \let\oldparagraph\paragraph
  \renewcommand{\paragraph}[1]{\oldparagraph{#1}\mbox{}}
\fi
\ifx\subparagraph\undefined\else
  \let\oldsubparagraph\subparagraph
  \renewcommand{\subparagraph}[1]{\oldsubparagraph{#1}\mbox{}}
\fi


\providecommand{\tightlist}{%
  \setlength{\itemsep}{0pt}\setlength{\parskip}{0pt}}\usepackage{longtable,booktabs,array}
\usepackage{calc} % for calculating minipage widths
% Correct order of tables after \paragraph or \subparagraph
\usepackage{etoolbox}
\makeatletter
\patchcmd\longtable{\par}{\if@noskipsec\mbox{}\fi\par}{}{}
\makeatother
% Allow footnotes in longtable head/foot
\IfFileExists{footnotehyper.sty}{\usepackage{footnotehyper}}{\usepackage{footnote}}
\makesavenoteenv{longtable}
\usepackage{graphicx}
\makeatletter
\def\maxwidth{\ifdim\Gin@nat@width>\linewidth\linewidth\else\Gin@nat@width\fi}
\def\maxheight{\ifdim\Gin@nat@height>\textheight\textheight\else\Gin@nat@height\fi}
\makeatother
% Scale images if necessary, so that they will not overflow the page
% margins by default, and it is still possible to overwrite the defaults
% using explicit options in \includegraphics[width, height, ...]{}
\setkeys{Gin}{width=\maxwidth,height=\maxheight,keepaspectratio}
% Set default figure placement to htbp
\makeatletter
\def\fps@figure{htbp}
\makeatother
% definitions for citeproc citations
\NewDocumentCommand\citeproctext{}{}
\NewDocumentCommand\citeproc{mm}{%
  \begingroup\def\citeproctext{#2}\cite{#1}\endgroup}
\makeatletter
 % allow citations to break across lines
 \let\@cite@ofmt\@firstofone
 % avoid brackets around text for \cite:
 \def\@biblabel#1{}
 \def\@cite#1#2{{#1\if@tempswa , #2\fi}}
\makeatother
\newlength{\cslhangindent}
\setlength{\cslhangindent}{1.5em}
\newlength{\csllabelwidth}
\setlength{\csllabelwidth}{3em}
\newenvironment{CSLReferences}[2] % #1 hanging-indent, #2 entry-spacing
 {\begin{list}{}{%
  \setlength{\itemindent}{0pt}
  \setlength{\leftmargin}{0pt}
  \setlength{\parsep}{0pt}
  % turn on hanging indent if param 1 is 1
  \ifodd #1
   \setlength{\leftmargin}{\cslhangindent}
   \setlength{\itemindent}{-1\cslhangindent}
  \fi
  % set entry spacing
  \setlength{\itemsep}{#2\baselineskip}}}
 {\end{list}}
\usepackage{calc}
\newcommand{\CSLBlock}[1]{\hfill\break\parbox[t]{\linewidth}{\strut\ignorespaces#1\strut}}
\newcommand{\CSLLeftMargin}[1]{\parbox[t]{\csllabelwidth}{\strut#1\strut}}
\newcommand{\CSLRightInline}[1]{\parbox[t]{\linewidth - \csllabelwidth}{\strut#1\strut}}
\newcommand{\CSLIndent}[1]{\hspace{\cslhangindent}#1}

\usepackage{algpseudocode}
\usepackage{algorithm}
\usepackage[noblocks]{authblk}
\usepackage[giveninits=false, uniquename=false]{biblatex}
\usepackage{mathtools}
\renewcommand*{\Authsep}{, }
\renewcommand*{\Authand}{, }
\renewcommand*{\Authands}{, }
\renewcommand\Affilfont{\small}
\DeclareMathOperator*{\argmin}{arg\,min}
\makeatletter
\@ifpackageloaded{caption}{}{\usepackage{caption}}
\AtBeginDocument{%
\ifdefined\contentsname
  \renewcommand*\contentsname{Table of contents}
\else
  \newcommand\contentsname{Table of contents}
\fi
\ifdefined\listfigurename
  \renewcommand*\listfigurename{List of Figures}
\else
  \newcommand\listfigurename{List of Figures}
\fi
\ifdefined\listtablename
  \renewcommand*\listtablename{List of Tables}
\else
  \newcommand\listtablename{List of Tables}
\fi
\ifdefined\figurename
  \renewcommand*\figurename{Figure}
\else
  \newcommand\figurename{Figure}
\fi
\ifdefined\tablename
  \renewcommand*\tablename{Table}
\else
  \newcommand\tablename{Table}
\fi
}
\@ifpackageloaded{float}{}{\usepackage{float}}
\floatstyle{ruled}
\@ifundefined{c@chapter}{\newfloat{codelisting}{h}{lop}}{\newfloat{codelisting}{h}{lop}[chapter]}
\floatname{codelisting}{Listing}
\newcommand*\listoflistings{\listof{codelisting}{List of Listings}}
\usepackage{amsthm}
\theoremstyle{definition}
\newtheorem{definition}{Definition}[section]
\theoremstyle{plain}
\newtheorem{theorem}{Theorem}[section]
\theoremstyle{remark}
\AtBeginDocument{\renewcommand*{\proofname}{Proof}}
\newtheorem*{remark}{Remark}
\newtheorem*{solution}{Solution}
\newtheorem{refremark}{Remark}[section]
\newtheorem{refsolution}{Solution}[section]
\makeatother
\makeatletter
\makeatother
\makeatletter
\@ifpackageloaded{caption}{}{\usepackage{caption}}
\@ifpackageloaded{subcaption}{}{\usepackage{subcaption}}
\makeatother
\ifLuaTeX
  \usepackage{selnolig}  % disable illegal ligatures
\fi
\usepackage{bookmark}

\IfFileExists{xurl.sty}{\usepackage{xurl}}{} % add URL line breaks if available
\urlstyle{same} % disable monospaced font for URLs
\hypersetup{
  pdftitle={Aggregating supply and demand with incomplete information and endogenous information acquisition (work in progress)},
  pdfauthor={Douglas K. G. Araujo},
  colorlinks=true,
  linkcolor={blue},
  filecolor={Maroon},
  citecolor={Blue},
  urlcolor={Blue},
  pdfcreator={LaTeX via pandoc}}

\title{Aggregating supply and demand with incomplete information and
endogenous information acquisition (work in progress)\thanks{This work
represents my opinion and not necessarily that of the BIS. An early
version of the manuscript circulated as `Information acquisition in
financial markets through artificial intelligence'.}}


  \author{Douglas K. G. Araujo}
            \affil{%
                  Bank for International
Settlements, douglas.araujo@bis.org
              }
      
\date{}
\begin{document}
\maketitle
\begin{abstract}
An economy is modelled as a simultaneous aggregation of multiple
bilateral interactions under incomplete information and with strategic
complementarities between supply and demand preferences, allowing for
arbitrary consumption function types and constraints. This view
microfounds the state of the economy with how well agents aticipate one
another's preferences, eg a output gap occurs when the weighted sum of
all agents' over- or underestimation of each other's preferences is not
neutral. Given the role of information about others' actions, the model
is extended with endogenous information acquisition, enabling a flexible
analysis of the role of information in the economy. This tooling is used
to study the macroeconomic impact of artificial intelligence (AI) models
that jointly considers its information aspect, strategic
complementarities and supply chain considerations. If AIs are widely
used to increase information precision across the board, the economy (is
mode efficient). But if the precision of AI is unequal then (society is
worse off if the largest players have access to better technology). JEL
codes: (D82, G14).
\end{abstract}

\section{Introduction}\label{introduction}

This paper microfounds the state of the economy from individual-level
demand and supply drivers, based on characteristics of each agent and on
an stochastic component, and on strategic complmentarities between pairs
of agents. The goal of this model is to offer a flexible alternative to
macroeconomic models that is built on the multiple bilateral strategic
interactions under incomplete information, and let the interaction graph
and more broadly, economic activity, follow from these two basic
building blocks of supply and demand drivers and strategic bilateral
interactions.

The model allows a natural breakdown of each of demand or supply
contributions to economic activity into a portion that is due to
fundamentals and another one due to strategic uncertainty. For example,
part of supply of labour might be associated to fundamentals such as
demographics and expected returns to education, but a part of these
dynamics can also be associated to the evolution of expected (by
households) demand of labour by firms; this can decrease if households
think firms will automate more, eg after the advent of popular ChatGPT
in late 2022. By the same token, firms' demand for labour reflects
fundamentals such as existing technology but also how much they expect
households to supply labour. If firms think households think firms would
automate more, then firms could think households would anticipate this
and (given certain returns to education) build up intellectual capital
instead of going into the labour market, and then firms could thus
choose to automate because they anticipate lower supply of labour.
Following the celebrated result in the global games literature (Carlsson
and Van Damme (1993), Morris and Shin (2003)), reasoning on higher-order
beliefs can be summarised by simpler and more effective forms of
reasoning such as threshold strategies.

One application of this model is to shed light on the role of AI as a
factor in economic activity. The model is flexible enough to be fit to
various types of microdata (firms only, firms and employees/consumers,
etc).

\subsection{Related literature}\label{related-literature}

\subsubsection{Strategic complementarities and
macroeconomics}\label{strategic-complementarities-and-macroeconomics}

Earlier efforts include Diamond (1982), who model a barter economy of
bilateral connections where transactions take place after a costly
search, and Cooper and John (1988), who analyse strategic
complementarities that result from production functions, demand
functions and matching technologies in a imperfectly competitive
heterogenous economy, but their model does not feature incomplete
information. More recently, Fernandez, Rudov, and Yariv (2022) model A
broad review of the role of incomplete information in macroeconomic
settings is provided at Angeletos and Lian (2016). Separately, Galeotti
et al. (2010) extend incomplete information to networks, modelling
uncertainty about the identity and count of neighbours in a network.
Supermodularity theory is also a flexible way to associate uncertainty
about the states of the world with the concept of risk aversion (Quiggin
and Chambers (2006)).

\subsubsection{Endogenous information
acquisition}\label{endogenous-information-acquisition}

Hellwig and Veldkamp (2009) model the situation where firms competing
endogenously acquire information to improve precision of their demand
forecasting. Farboodi and Veldkamp (2020)'s model distinguishes between
processing information to better understand fundamentals as opposed to
forecasting the demand of other investors.

\subsubsection{Effects of information on the
economy}\label{effects-of-information-on-the-economy}

Acemoglu (2024), in contrast, estimates AI will not have such a
significant impact.

\section{Static setup}\label{static-setup}

This section describes a static version of the game to present the main
building blocks and fix intuition.

\subsection{Economic agents and their
interactions}\label{economic-agents-and-their-interactions}

An economy comprises individual agents \(i = 1, \dots, N\). These agents
are for now described as ``firms'', an interpretation that is relaxed in
Section~\ref{sec-interpretation}. Firms transact bilaterally with one
another through the asymmetric matrix \(J \in \{0, 1\}^{N \times N}\),
where element \(j_{ij} = 1\) means that \(i\) buys some unspecified good
or service from \(j\) and \(0\) denotes the inexistence of this
relationship.

The element \(j_{ij}\) results from the combination of \(i\)'s demand
for \(j\)'s product, \(d_{ij} \geq 0\) and \(j\)'s preference or
preparedness for selling products to \(i\), \(s_{ji} \geq 0\). Each firm
spawns with a demand and a supply profile,
\(d_i, s_i \in \mathbb{B}^N\), and they always known their own profiles.
Define \(\theta_{ij} = d_{ij} s_{ji}\) to be the common payoff for \(i\)
and \(j\) if \(i\) buys from \(j\); if the transaction does not take
place, the payoff is \(0\).\footnote{Cases like that of repugnant
  transactions (Roth (2007)), where either \(d_{ij}\) or \(s_{ji}\) are
  negative (or both) can be dealt with by this model but is not an
  essential component, so is not further discussed except to say that if
  a possibility exists that both demand and supply are negative, then
  \(\theta_{ij} = \text{max}(0, d_{ij} s_{ji})\).} Collecting the demand
and supply elements in their own matrices \(D\) and \(S\), then
\(\theta = D \odot S^T\), with \(\odot\) representing the element-wise
Hadamard multiplication. Using \(N=2\) to illustrate these values,

\[
D = 
\begin{bmatrix}
d_{11} & d_{12} \\
d_{21} & d_{22}
\end{bmatrix}
,
S = 
\begin{bmatrix}
s_{11} & s_{12} \\
s_{21} & s_{22}
\end{bmatrix}
,
\theta = 
\begin{bmatrix}
d_{11}s_{11} & d_{12}s_{21} \\
d_{21}s_{12} & d_{22}s_{22}
\end{bmatrix}.
\]

Then, bilateral transactions take place according to a map
\(C : \mathbb{R}^{N \times N} \to \{0, 1\}^{N \times N}\) that is
elementwise non-decreasing in \(\theta\), ie
\(C(\theta')_{ij} \geq C(\theta)_{ij}\) if: \[
\theta' = \begin{cases} \theta_{mn}^{'} > \theta_{mn} & m=i, n=j\\ \theta_{mn} & \text{otherwise} \end{cases}.
\]

This consumption function \(C\) is very general, but once it is applied
on the payoff matrix, it yields the bilateral transaction matrix
\(J = C(\theta)\). Because \(C\) is elementwise non-decreasing in
\(\theta\), whenever there is any form of constraint (budget,
technological or otherwise), firms will choose to demand only the
bilateral transactions they rank highest, and to supply only to those
clients who the firm as a supplier value the most. So it is useful
regardless of the shape of \(C\) to know how each firm values their
demand for each bilateral transaction relative to alternatives, and the
same from the supply perspective. A useful way to do this is with a
softmax function over the \(i\)th row of \(\theta\) for the demand:

\begin{equation}\phantomsection\label{eq-softmaxrow}{
\tilde{d}_{ij} = \frac{e^{\theta_{ij}}}{\sum_{k=1}^N e^{\theta_{ik}}},
}\end{equation}

and similarly using \(\theta\)'s columns for the supply interactions:

\begin{equation}\phantomsection\label{eq-softmaxcol}{
\tilde{s}_{ji} = \frac{e^{\theta_{ji}}}{\sum_{k=1}^N e^{\theta_{ki}}}.
}\end{equation}

For exposition, take \(C\) to be a function that matches buyers and
sellers according to a ``budget'' that limits how many matches they
might attempt. Formally, each \(i\) has a budget of
\(K_i^d \leq N-1 \in \mathbb{N}\) maximum bilateral purchase
transactions and an arbitrary budget of
\(K_i^s \leq N-1 \in \mathbb{N}\) bilateral sales transactions for
simplicity. This means that \(i\) will choose to buy from the top
\(K_i^d\) highest values in row \(i\) of \(\tilde{D}\), and sell to the
\(K_i^s\) highest values in column \(i\) of \(\tilde{S}\). In other
words, each agent will choose to act only on their own most advantageous
purchases as valued by Equation~\ref{eq-softmaxrow} and
Equation~\ref{eq-softmaxcol} respectively. Note that \(i\) can prefer to
interact with itself; see Section~\ref{sec-interpretation} for comments
on the interpretation of these cases.

These decisions to actually demand a product or supply it are stored in
vectors \(a_i^d, a_i^s\) for each firm's demand and supply actions.
These vectors span the space of possible strategy profiles for the
demand and supply actions, respectively. Based on the previously
discussed top-\(K\) rule, define the matrices \(A^d\) and \(A^s\) to
collect, respectively, the \(\{0, 1\}\) decision of each firm \(i\) to
demand or to supply goods to firm \(j\). When both the demand decision
\(a_{ij}^d\) and supply decision \(a_{ij}\) are \(1\), then \(J_{ij}\)
is also \(1\); otherwise it is \(0\). Expanding on the \(N=2\) example
above, and assuming \(K_i^d=1 \; \forall i\), \(K_i^s=2 \; \forall i\),
with an arbitrary ordering that assumes that transaction with another
firm is more valuable than with oneself:

\[
\tilde{D} = 
\begin{bmatrix}
\tilde{d}_{11} & \tilde{d}_{12} \\
\tilde{d}_{21} & \tilde{d}_{22}
\end{bmatrix}
,
\tilde{S} = 
\begin{bmatrix}
\tilde{s}_{11} & \tilde{s}_{12} \\
\tilde{s}_{21} & \tilde{s}_{22}
\end{bmatrix}
,
A^d = 
\begin{bmatrix}
0 & 1 \\
1 & 0
\end{bmatrix}
,
A^s = 
\begin{bmatrix}
1 & 1 \\
1 & 1
\end{bmatrix}
,
J = 
\begin{bmatrix}
0 & 1 \\
1 & 0
\end{bmatrix}.
\]

The values above can also be summarised in an outcomes matrix \(O\),
where each element \(o_{ij}\) corresponds to the outcome of each
bilateral transaction between \(i\) and \(j\). Beyond gross payoffs
\(\theta_{ij}\), which accrue to both counterparties if both engage in
that transaction, to provide some more generality to the model consider
also that actually positioning oneself for demanding or supplying goods
or services to another firm entails non-negative costs
\(T^d, T^s \geq 0\), with elements \(t_{ij}^{z \in \{d, s\}}\). They
might represent, for example, distance betwen \(i\) and \(j\),
investment in advertising tailored to \(j\) tastes, search costs, sales
efforts, etc. (As for the other building blocks, there is ample
flexibility in how the costs are estimated in practice.) Thus the
outcomes are:

\begin{equation}\phantomsection\label{eq-outcomes}{
O = 
\begin{bmatrix}
-(a_{11}^d t_{11}^d) + (a_{11}^d d_{11} s_{11} a_{11}^s) - (a_{11}^s t_{11}^s) & -(a_{12}^d t_{12}^d) + (a_{12}^d d_{12} s_{21} a_{21}^s) - (a_{21}^s t_{21}^s)\\
-(a_{21}^d t_{21}^d) + (a_{21}^d d_{21} s_{12} a_{12}^s) - (a_{21}^s t_{21}^s) & -(a_{22}^d t_{22}^d) + (a_{22}^d d_{22} s_{22} a_{22}^s) - (a_{22}^s t_{22}^s)
\end{bmatrix}.
}\end{equation}

Each element \(o_{ij}\) of Equation~\ref{eq-outcomes} has four possible
values. When no firm decides to demand or supply that particular
transaction, the outcome is \(0\). When at most one firm engages, the
outcome is its demand or supply cost. And when both agents coordinate,
\(o_{ij}\) is defined as the combined demand and supply driver. In the
current example, \(O=\big[\begin{smallmatrix}
-t_{11}^s & d_{12} s_{21} - t_{12}^d - t_{21}^s \\
d_{21} s_{12} -t_{21}^d - t_{12}^s & - t_{22}^s
\end{smallmatrix}\big]\).

The outcomes of the bilateral transactions speak to the overall value
added (or subtracted). But what each firm cares about is their own
payoff from each transaction, which is defined as the combined value of
the transaction, \(\theta_{ij}\), minus costs incurred in positioning
themselves for demanding (\(t_{ij}^d\)) or supplying (\(t_{ij}^s\)) that
transaction (or both demanding and supplying if \(i=j\)). Aggregating
the payoffs at the firm level results in vector \(p = (p_i)_{i=1}^N\),
with:

\begin{equation}\phantomsection\label{eq-firmlevelpayoff}{
p_i = \sum_{j=1}^N \underbrace{a_{ij}^d (d_{ij} s_{ji} a_{ji}^s - t_{ij}^d)}_{\text{Net payoff from $i$'s demand to match with $j$}} + \underbrace{a_{ij}^s (a_{ji}^d d_{ji} s_{ij} - t_{ij}^s)}_{\text{Net payoff from $i$ supplying to $j$}}.
}\end{equation}

For firm 1 and firm 2 in the current example, this is:

\[
\begin{split}
p_1 = d_{12} s_{21} - t_{12}^d - t_{11}^s + d_{21} s_{12} - t_{12}^s, \\
p_2 = d_{21} s_{12} -t_{21}^d + d_{12} s_{21} - t_{21}^s - t_{22}^s.
\end{split}
\]

In this static game, firms first discover \(D\) and \(S\) and then
decide how much and from whom to demand or to supply in \(A^d, A^d\). In
a complete information benchmark, firms know all demand and supply
drivers and the budget function and constraints of each other firm.
Armed with this information, all firms can estimate \(A^d\) and \(A^s\)
for all firms, and thus all will know the implied \(P\) as well as the
payoff under any deviation from \(A^d\) or \(A^s\). This allows firms to
conclude that \((A^d(D), A^s(S))\) is the Nash equilibrium action
profile. Deviating unilaterally from that is not advantageous to any
particular firm because it will take demand or supply away from good
transactions into transactions that are not met correspondingly by the
counterparty supplying or demanding that good.

This economy abstracts away from important components of
``micro-to-macro'' aggregation (eg, Baqaee and Farhi (2018), Baqaee and
Farhi (2019)), such as a definition of elasticities of substitution.
This is to concentrate attention on the aspects of the model that are
more dependent on coordination between supply and demand.

\subsection{Incomplete information}\label{incomplete-information}

The key aspect of this model is that each firm knows only the
realisation of its own demand towards firm \(j\)'s products and its own
preference for supplying goods and services to all \(j\) firms. For all
other cases, ie firm \(j \neq i\)'s demand for \(k\)'s products and its
supply to firm \(k\) are only observed with a noisy signal.\footnote{Traditionally,
  games of incomplete information such as global games (Carlsson and Van
  Damme (1993), Morris and Shin (2003), Frankel, Morris, and Pauzner
  (2003)) model an additive noise strucutre, but for analytical
  convenience. This device continues to be used (Szkup and Trevino
  (2020)), but the literature also records other, more flexible forms of
  perturbation (eg, Yang (2015)).} Henceforth, variables with a hat are
estimated with some (possibly non-linear) noise by an investor, each
with a specific precision \(\tau_i(\cdot)\); for example,
\(\tau_1(\hat{d}_{21})\) the precision of firm \(1\)'s estimate of
\(d_{21}\). Continuing with investor \(i=1\) in the example above, the
game begins with the following information profile:

\[
\hat{D}^{(1)} = 
\begin{bmatrix}
d_{11} & d_{12} \\
\hat{d}_{21}^{(1)} & \hat{d}_{22}^{(1)}
\end{bmatrix}
,
\hat{S}^{(1)} = 
\begin{bmatrix}
s_{11} & s_{12} \\
\hat{s}_{21}^{(1)} & \hat{s}_{22}^{(1)}
\end{bmatrix}
,
\hat{\theta}^{(1)} = 
\begin{bmatrix}
d_{11}s_{11} & d_{12}\hat{s}_{21}^{(1)} \\
\hat{d}_{21}^{(1)} s_{12} & \hat{d}_{22}^{(1)} \hat{s}_{22}^{(1)}
\end{bmatrix}.
\]

The superscript in the hat variables indicates that it is firm \(1\)'s
signal of the other firm's demand or supply. To avoid cluttered
notation, this superscript is dropped when equations refer to the
running \(N=2\) example or when it is clear it refers to a generic firm
\(i\)'s signal, and will only be used to remove ambiguity.

Given the uncertainty about the actual demand or supply driver of other
firms, firm \(1\)'s views on \(\tilde{D}, \tilde{S}\) are given by
Equation~\ref{eq-tildeDS} below:

\begin{equation}\phantomsection\label{eq-tildeDS}{
\hat{\tilde{D}} = 
\begin{bmatrix}
\tilde{d}_{11} & \tilde{d}_{12} \\
\hat{\tilde{d}}_{21} & \hat{\tilde{d}}_{22}
\end{bmatrix}
,
\hat{\tilde{S}} = 
\begin{bmatrix}
\tilde{s}_{11} & \tilde{s}_{12} \\
\hat{\tilde{s}}_{21} & \hat{\tilde{s}}_{22}
\end{bmatrix}.
}\end{equation}

Note that this construction enables a generalisation of the incomplete
information setting where even the consumption function (including any
budget constraints) of the other firms are not necessarily known by
\(i\). The only requirement is that this function be elementwise
non-decreasing in the payoff. This brings more realism to the model in
the sense that in the real world, economic agents make decisions without
fully known how ``deep'' their pockets are, or whether or not they have
access to other sources of financing, etc. At the same time, this
construction does not add considerable complexity to the model. the
disadvantage is that applications where the budget constraint will be
studied, say in comparative statics would require a more explicit
specification of the consumption function.

Firm \(1\)'s view on the outcomes and its own expected payoff structure
reflects the information profile: the demand and supply that firm \(1\)
has towards other firms, \((d_{1j})_{j=1}^N\) and \((s_{1j})_{j=1}^N\),
as well as \(1\)'s own actions, \((a_{1j}^d)_{j=1}^N\) and
\((a_{1j}^s)_{j=1}^N\), are known by 1, whereas other firms' are
estimated. In the \(N=2\) example:

\begin{equation}\phantomsection\label{eq-iioutcomes}{
O = 
\begin{bmatrix}
-(a_{11}^d t_{11}^d) + (a_{11}^d d_{11} s_{11} a_{11}^s) - (a_{11}^s t_{11}^s) & -(a_{12}^d t_{12}^d) + (a_{12}^d d_{12} \hat{s}_{21} \hat{a}_{21}^s) - (a_{21}^s t_{21}^s)\\
-(a_{21}^d t_{21}^d) + (a_{21}^d d_{21} s_{12} a_{12}^s) - (a_{21}^s t_{21}^s) & -(a_{22}^d t_{22}^d) + (a_{22}^d d_{22} s_{22} a_{22}^s) - (a_{22}^s t_{22}^s)
\end{bmatrix}.
}\end{equation}

Equation~\ref{eq-iioutcomes} underscores the strategic uncertainty about
firm \(2\)'s demand and supply and also about their actions, since firm
\(1\) does not know firm \(2\)'s budget constraints, consumption or
supply form or any other information except the importance of the actual
payoff to determine a higher chance of a transaction occuring.

Adapting Equation~\ref{eq-firmlevelpayoff} to a scenario where firm
\(i\) calculates expected payoffs,

\begin{equation}\phantomsection\label{eq-iifirmlevelpayoff}{
\mathbb{E}[p_i] = a_{ii}^d (d_{ii} s_{ii} a_{ii}^s - t_{ii}^d) + a_{ii}^s (a_{ii}^d d_{ii} s_{ii} - t_{ii}^s) + \sum_{j \in \{1, \dots, N\}\setminus \{i\}} a_{ij}^d (d_{ij} \hat{s}_{ji} \hat{a}_{ji}^s - t_{ij}^d) + a_{ij}^s (\hat{a}_{ji}^d \hat{d}_{ji} s_{ij} - t_{ij}^s)
}\end{equation}

where the first term correspond to the own actions and
(self-)transaction values and the second term includes payoffs on which
firm \(i\) is uncertain. Going forward, to simplify notation collect the
first term in
\(p_{ii} = a_{ii}^d (d_{ii} s_{ii} a_{ii}^s - t_{ii}^d) + a_{ii}^s (a_{ii}^d d_{ii} s_{ii} - t_{ii}^s)\).
To be clear, the decision to demand and supply own transactions is
jointly determined along with all other alternatives, which are
uncertain. So the ``known'' terms here must be read as being conditional
on the received signals \(\hat{D}, \hat{S}\).

Given these monotone dependencies and the uncertainty about the
consumption function shape for other firms, each firm defines a
threshold strategy to decide \(a_{i}^d, a_{i}^s\), subject to their own
constraints. This threshold, which is specific to each firm' demand and
supply relationships, already incorporates any constraints at the
firm-level (such as budget limits or technological envelope).
Importantly, the threshold strategy also applies for own \(i\)-to-\(i\)
relationships, since they might also be constrained generically even if
the information for these particular matches is complete, or because of
the influence of other, estimated, relationships. The bilateral actions
taken by firm \(i\) will be defined by a pair of thresholds
\(d_{i}^*, s_{i}^*\). Specifically,

\begin{equation}\phantomsection\label{eq-thresholds}{
\begin{split}
a_{ij}^d = \mathbb{1}[\tilde{d}_{ij} \geq d_{i}^*] \\
a_{ij}^s = \mathbb{1}[\tilde{s}_{ij} \geq s_{i}^*],
\end{split}
}\end{equation}

with these thresholds representing the point of indifference for firms.
Putting together all these elements, the optimal thresholds would then
each equate the expected payoffs with the action costs, as in:

\begin{equation}\phantomsection\label{eq-indif}{
\begin{split}
\mathbb{E}[\hat{p}_{ij} \text{Pr}(s_{ji} \geq \hat{s}_{ji}) &| \hat{d}_{i}^*] = t_{ij}^d \\
\mathbb{E}[\hat{p}_{ij} \text{Pr}(d_{ij} \geq \hat{d}_{ij}) &| \hat{s}_{j}^*] = t_{ij}^s.
\end{split}
}\end{equation}

The left-hand side of both indifference equations comes from the
assumption held by each firm that all other firms would use threshold
strategies. Collect the individual thresholds into vectors of firm-level
demand and supply thresholds, named \(\mathbf{d}^*\) and
\(\mathbf{s}^*\) respectively. Definition~\ref{def-optmthresh} offers
more details on these objects.

\begin{definition}[Optimal threshold
vectors]\protect\hypertarget{def-optmthresh}{}\label{def-optmthresh}

Given demand and supply signals \(\hat{D}\) and \(\hat{S}\) and their
precisions
\(\tau = \bigcup_{Z \in \{\hat{D}, \hat{S}\}} (\tau_{ij}(Z))_{i=1, j=1}^N\),
there are vectors of optimal thresholds for each of demand and supply,
respectively,
\(\mathbf{d}^* = (d_i^*(\hat{D}, \hat{S}, \tau))_{i=1}^N \in [0, 1]^N\)
and
\(\mathbf{s}^* = (s_i^*(\hat{D}, \hat{S}, \tau))_{i=1}^N \in [0, 1]^N\).
Each firm \(i\) uses the threshold \(d_i^*\) to evaluate all the
possibilities for demand and for supply matching, after they have been
transformed into the \([0, 1]\) domain by the softmax function in
Equation~\ref{eq-softmaxrow} and Equation~\ref{eq-softmaxcol}. Each pair
\((d_i^*, s_i^*)\) optimises firm \(i\)'s expected payoff
\(\sum_{j=1}^N p_(ij)\) subject to its constraints.

\emph{Remark 1.} Higher thresholds for firm \(i\) make it less likely
that \(i\) will coordinate.

\emph{Remark 2.} For firm \(i\), \(d_i^*\) is not necessarily equal to
\(s_i^*\), but they can be co-determined within each firm. For example,
a supermarket firm (which has a low cost of supplying to a broad base of
potential clients, and expectats that many clients would correspond with
their demand to its assortiment of products) has a low supply threshold
\(s_i^*\), but it also must have a low \(d_i^*\) lest it would only
purchase very specific goods, which would not be fitting for a
supermarket.

\emph{Remark 3.} For brevity, the term in parenthesis indicating a
dependence on the demand and supply signals and and on the signal
precision might be ommited but it should always be understood that these
vectors are defined with the complete \(\hat{D}\) and \(\hat{S}\)
matrices.

\emph{Example 1.} Equation~\ref{eq-thresholds} demonstrates the use of
the optimal threshold in the \(N=2\) example.

\emph{Example 2.} Suppose
\(\mathbf{d}^{*} = \mathbf{s}^{*} = (0)_{N}*\), ie, both demand and
supply thresholds are 0 for all firms. In this case, all firms would act
on their supply and demand, and form connections whenever these match an
opposing action by counterparty. The possibility of equality as opposed
to strict inequality (see again Equation~\ref{eq-thresholds}) would make
firms always act even in cases where their original demand (or supply)
preferences were \(0\).

\emph{Example 3.} Conversely, if
\(\mathbf{d}^* = \mathbf{s}^* = (1)_N*\), ie, both demand and supply
thresholds are 1 for all firms, then no firm would act on their demand
or supply, except when these are all concentrated in a single bilateral
relationship.\footnote{I posit informally (but will not further prove or
  discuss) that such a situation would make the economy degenerate to
  autarky, since it would lead all firms to place all demand and supply
  efforts only on themselves expecting other firms to do the same.}

\end{definition}

\subsection{Coordination as a monotone supermodular
game}\label{coordination-as-a-monotone-supermodular-game}

Now that the optimal thresholds are defined and this coordination stage
has all the building blocks, it is important to show that the present
model is a monotone supermodular games (as in Van Zandt and Vives
(2007)) and then use this fact to present a theorem defining the
equilibrium for the model so far (with exogenous information precision
profiles). Supermodular games, introduced by Topkis (1979) and further
studied by Milgrom and Roberts (1990), various papers by Vives are
classes of games that rely on monotonicity properties and on the
existence of strategic complementarities to find equilibria.\footnote{Amir
  (2005), Vives (2005), Vives (2007) and Vives and Vravosinos (2024) are
  useful surveys of this literature.}

One specific challenge is to show that the high-dimensional, non-linear
environment does not interfere with the monotonicity requirement.
\ldots.. In contrast, Morris, Shin, and Yildiz (2016) offer solutions
for classes of models that do not respect monotonicity in types.

Based on this connection, Theorem~\ref{thm-optmthresh} establishes their
existence and uniqueness.

\begin{theorem}[Existence and uniqueness of optimal
thresholds]\protect\hypertarget{thm-optmthresh}{}\label{thm-optmthresh}

There exists precision configurations \(\tau^*\) for which a unique
Bayesian Nash equilibrium surviving iterated deletion of strictly
dominated strategies exist, with firms using thresholds
\(\mathbf{d}^*(\tau_{ij}(\hat{D}, \hat{S}))\) and
\(\mathbf{s}^*(\tau_{ij}(\hat{D}, \hat{S}))\) to decide on bilateral
actions.

\end{theorem}

\begin{proof}
(Under construction) This proof is based on showing that the game is a
monotone supermodular game, as in Van Zandt and Vives (2007), and use
that result to show there is an equilibrium (see definitions
Definition~\ref{def-supermodular} and
Definition~\ref{def-monsupermodular} in the Annex). From Example
2.6.2(a) in Topkis (1998), the supermodularity of the payoff \(\pi_i\)
on own actions \(a_{ij}^{z \in \{d, s\}}\) comes from the fact that it
is a totally ordered set. The supermodularity (increasing differences)
on other firms' actions is demonstrated below. Writing
\(\pi_i((a_i^d, a_i^s),(a_{-i}^d, a_{-i}^s), \theta)\) to represent
\(i\)'s payoff when it takes actions \(a_i^d, a_i^s\), other players
take actions \(a_{-i}^d, a_{-i}^s\) and the demand and supply shocks are
summarised in \(\theta\), then strategic complementarity in others'
actions entails
\(\pi_i((1,1), (0,0), \theta) - \pi_i((0,0), (0,0), \theta) \leq \pi_i((1,1), (1,1), \theta) - \pi_i((0,0), (1,1), \theta)\):

\[
\sum_{j=1}^N -t_{ij}^d -t_{ij}^s \leq \sum_{j=1}^N \theta_{ij} + \theta_{ji} - t_{ij}^d - t_{ij}^s,
\]

since all elements of \(\theta\) are non-negative.

The same argument holds for each of the demand and supply actions
separately, also when crossed with counterparties' supply or demand
actions.
\end{proof}

Theorem~\ref{thm-optmthresh} establishes that the firms in the economy
reach an equilibrium.

\subsection{Interpretation}\label{sec-interpretation}

\subsubsection{Types of agents}\label{types-of-agents}

The economic agents, described above as ``firms'', but this description
is more general. For example, if \(x_i\) is a vector of characteristics
of each \(i\), which do not have to be unique, then one element of the
vector \(x_i\) can represent the industry or the location of \(i\). This
would already allow the construction of metrics of interest to
applications that focus on certain subsets of firms, for example. One
interesting application could be for example to set that firms whose
characteristic vector include them in the ``retail'' sector would tend
to try to match a much broader subset of agents then specialised firms
in high-end supply chains, which usually have only a few clients.

In fact, the entities can be more generally taken to be both firms and
natural persons; even one or more government entities could be economic
agents. As long as \(x_i\) contains enough information to differentiate
them, it is possible to pool together all sorts of economic agents in
this economy if the theoretical or empirical application so requires. Of
course, constraints (and interpretation) would need to be well-thought
if the space of individuals encompasses such rich representations. For
example, if some \(i\) are natural persons, then a reasonable constraint
to impose of them that most firms will not have is that they will
``sell'' their product (labour) to only one firm. Or, for those \(i\)
that are government entities, the product that they ``sell'' are
certificates of paid taxes (naturally, against proceeds that are
interpreted as tax) while their purchases are the government
expenditures.

\subsubsection{Within-agent interaction}\label{within-agent-interaction}

An agent might prefer to demand its own product (think of Apple
employees using Macbooks and iPhones) and also might prefer to sell to
itself: for example, if demand for a product is weak it might end up
building up stock.

\subsubsection{State of the economy}\label{state-of-the-economy}

Similar to Angeletos and Lian (2016) (see section 3.1), in this model
the state of the economy is seen as the distribution of actions
throughout the economy. However, with this model it is possible to go
one step further and construct different subsets and aggregations of the
economy. For example, the complete economy payoff is
\(\sum_{i=1}^N \sum_{j=1}^N p_{ij}\), while wages are
\(\sum_{i=1}^N \sum_{j \in \{\text{Natural persons}\}} p_{ij}\). Labour
productivity is given by
\(\sum_{i=1}^N \sum_{j=1}^N p_{ij} / \sum_{i=1}^N \sum_{j \in \{\text{Natural persons}\}} J_{ij}\),
while cost efficiency of the economy is
\(\sum_{z \in \{D, S\}}\sum_{i=1}^N \sum_{j=1}^N t_{ij}^z a_{ij}^z\).
Firm failures result from counting all the firms for whom the net payoff
is negative.

\subsubsection{Frictions}\label{frictions}

In this model, output gaps are explained by lack of coordination from
incomplete information. A similar friction was introduced by Diamond
(1982), who add trade (or more generally, search) frictions in settings
where multiple people must transact simultaneously and there is no
Walrasian auctioneer to ensure the market clears frictionless. But in
contrast to that paper, the frictions in the current model all come from
the necessity of anticipating the demand and supply of other agents.

\subsection{Endogenous information
acquisition}\label{endogenous-information-acquisition-1}

This section now expands the theme from the previous one, incorporating
a prior stage where all firms are able to negotiate with firms that sell
information technology. The possibility that firms endogenously choose
the level of information to address the strategic uncertainty has been
shown to change equilibrium selection both in theory and in experimental
settings. For example, Szkup and Trevino (2024) find that agents tend to
select themselves more by the amount of information acquired than by the
investment decisions at a coordination phase, compared to the same model
but with exogenous information precision (Szkup and Trevino (2020)).

The literature on endogenous information acquisition is vast (see for
example Amir and Lazzati (2016), Colombo, Femminis, and Pavan (2014),
Szkup and Trevino (2015), Yang (2015) and others). In a beauty-contest
setting, Myatt and Wallace (2012) study allocation of attention to
information and find that the degree of strategic complementarity is
negatively related to the number of signals watched by individuals, who
rely increasingly in a public signal. Szkup and Trevino (2024) find that
players subjectively select different levels of information, beyond
rationalisable choiced.

Particularly important results in endogenous information aquisition are
due to Colombo, Femminis, and Pavan (2014).

An important application of endogenous information aquisition is the
study of breakthrough improvements in information processing technology,
in particular more recently the role of AI in coordination amongst
agents. In fact, related papers show that firm investment in AI is
heterogenous and has wide-ranging implications for the firm. For
example, firms investing more in AI usage tend to reshape their labour
composition and organisation (Babina et al. (2023)), indicating the firm
is positioning itself to better serve existing markets (concentrating
more their \(s_j\) to better match latent demand \(d_{ij}\)) or to
acquire new, underserved markets, as reinforced by empirical evidence
(Babina et al. (2024)). After discussing how the present model changes
in response to endogenous information aquisition, this paper uses these
results to elaborate on the influence of AI on the economy in
Section~\ref{sec-AI}.

\section{Application: effects of artificial intelligence}\label{sec-AI}

This model allows the study of different channels by which AI might
impact macroeconomic outcomes. These are:

\begin{itemize}
\item
  \textbf{Improving the precision of economic signals.} \(\tau_{ij}\)
\item
  \textbf{Increasing product innovation.}
\item
  \textbf{Lowering the costs associated with demanding or supplying
  specific products.} \(T^d, T^s\) Allows a greater number of bilateral
  relationships to take place.
\item
  \textbf{Increasing labour automation.} This case requires the
  inclusion of agents that span both firms and natural persons.
\end{itemize}

\subsection{AI improving signal
precision}\label{ai-improving-signal-precision}

\subsection{AI increases product
innovation}\label{ai-increases-product-innovation}

Babina et al. (2024) show that greater investment in AI leads to
enhanced product innovation in firms. They find that this effect leads
to boosted growth, contributing to concentration and to the growth of
superstar firms.

\subsection{AI lowering costs}\label{ai-lowering-costs}

The costs incurred by firm \(i\) to demand goods are broken down as
\(t_{ij}^d = t_0 + t_i + \omega_{ij} \geq 0\), with a similar definition
for supply cost considerations.

\subsection{AI increasing labour
automation}\label{ai-increasing-labour-automation}

\section{Dynamic setting (under
development)}\label{dynamic-setting-under-development}

Generally speaking, equilibria in monotone strategies in dynamic
settings are more restrictive than in the static version, as they
require monotone best replies(Mensch (2020)).

Previous work exploring coordination in dynamic settings include Amir
(1996) and Datta, Reffett, and Woźny (2018).

\section{(Tentative) conclusions}\label{tentative-conclusions}

The direct use of lattice-theoretical tools in macroeconomics has a rich
tradition even if it does not currently inform mainstream models. But as
the focus of policymakers shifts from representative agents to
understanding heterogeneity and microfounding models, and especially as
the way agents acquire and process their information sets to interact
with other agents under strategic complementarity settings, supermodular
game analysis might offer insights in macroeconomics as it has in
industrial organisation, microeconomics and other fields (Amir (2005),
Amir (2018)). Supermodular analysis is based on the intutive insight
that if an objective function to be maximised reacts in a complementary
way between its endogenous and exogenous inputs, then the optimal value
of the endogenous variables increases with the exogenous values. The
fact that these methods dispense with seemingly superfluous assumptions
eg on functional form of utility functions and instead rely on
monotonicity constraints further lends it flexibility even for
non-linear analysis.

As for compatitive statics related to AI, AI is a palpable
transformation to information processing and use, promising deep changes
even in the societal structure. Even before the current wave of public
awareness after the release of ChatGPT in November 2022, Korinek and
Stiglitz (2018) pointed to the potential effects of AI on the economy as
a whole.

But for all its importance, AI is not the only information-relevant
transformation. Over the last decades, innovations such as the
long-distance telephone, internet, and early machine learning models all
have increased the capabilities of transforming information. And in the
other direction, greater awareness and effective privacy at the consumer
level has driven significant economic impacts (Doerr et al. (2023),
Aridor and Che (2024)). A flexible model is needed to study these
wide-ranging transformation.

\section{References}\label{references}

\phantomsection\label{refs}
\begin{CSLReferences}{1}{0}
\bibitem[\citeproctext]{ref-acemoglu2024simple}
Acemoglu, Daron. 2024. {``The Simple Macroeconomics of AI.''} National
Bureau of Economic Research.

\bibitem[\citeproctext]{ref-amir1996continuous}
Amir, Rabah. 1996. {``Continuous Stochastic Games of Capital
Accumulation with Convex Transitions.''} \emph{Games and Economic
Behavior} 15 (2): 111--31.

\bibitem[\citeproctext]{ref-amir2005supermodularity}
---------. 2005. {``Supermodularity and Complementarity in Economics: An
Elementary Survey.''} \emph{Southern Economic Journal} 71 (3): 636--60.

\bibitem[\citeproctext]{ref-amir2018supermodularity}
---------. 2018. {``Supermodularity and Monotone Methods in
Economics.''} \emph{Economic Theory}. Springer.

\bibitem[\citeproctext]{ref-AMIR2016684}
Amir, Rabah, and Natalia Lazzati. 2016. {``Endogenous Information
Acquisition in Bayesian Games with Strategic Complementarities.''}
\emph{Journal of Economic Theory} 163: 684--98.
https://doi.org/\url{https://doi.org/10.1016/j.jet.2016.03.005}.

\bibitem[\citeproctext]{ref-angeletos2016incomplete}
Angeletos, G-M, and Chen Lian. 2016. {``Incomplete Information in
Macroeconomics: Accommodating Frictions in Coordination.''} In
\emph{Handbook of Macroeconomics}, 2:1065--1240. Elsevier.

\bibitem[\citeproctext]{ref-aridor2024privacy}
Aridor, Guy, and Yeon-Koo Che. 2024. {``Privacy Regulation and Targeted
Advertising: Evidence from Apple's App Tracking Transparency.''}

\bibitem[\citeproctext]{ref-babina2023firm}
Babina, Tania, Anastassia Fedyk, Alex X He, and James Hodson. 2023.
\emph{Firm Investments in Artificial Intelligence Technologies and
Changes in Workforce Composition}. Vol. 31325. National Bureau of
Economic Research.

\bibitem[\citeproctext]{ref-babina2024artificial}
Babina, Tania, Anastassia Fedyk, Alex He, and James Hodson. 2024.
{``Artificial Intelligence, Firm Growth, and Product Innovation.''}
\emph{Journal of Financial Economics} 151: 103745.

\bibitem[\citeproctext]{ref-baqaee2018macroeconomics}
Baqaee, David Rezza, and Emmanuel Farhi. 2018. {``Macroeconomics with
Heterogeneous Agents and Input-Output Networks.''} National Bureau of
Economic Research.

\bibitem[\citeproctext]{ref-baqaee2019jeea}
---------. 2019. {``Jeea-Fbbva Lecture 2018: The Microeconomic
Foundations of Aggregate Production Functions.''} \emph{Journal of the
European Economic Association} 17 (5): 1337--92.

\bibitem[\citeproctext]{ref-carlsson1993global}
Carlsson, Hans, and Eric Van Damme. 1993. {``Global Games and
Equilibrium Selection.''} \emph{Econometrica: Journal of the Econometric
Society}, 989--1018.

\bibitem[\citeproctext]{ref-colombo2014information}
Colombo, Luca, Gianluca Femminis, and Alessandro Pavan. 2014.
{``Information Acquisition and Welfare.''} \emph{The Review of Economic
Studies} 81 (4): 1438--83.

\bibitem[\citeproctext]{ref-cooper1988coordinating}
Cooper, Russell, and Andrew John. 1988. {``Coordinating Coordination
Failures in Keynesian Models.''} \emph{The Quarterly Journal of
Economics} 103 (3): 441--63.

\bibitem[\citeproctext]{ref-datta2018comparing}
Datta, Manjira, Kevin Reffett, and Łukasz Woźny. 2018. {``Comparing
Recursive Equilibrium in Economies with Dynamic Complementarities and
Indeterminacy.''} \emph{Economic Theory} 66: 593--626.

\bibitem[\citeproctext]{ref-diamond1982aggregate}
Diamond, Peter A. 1982. {``Aggregate Demand Management in Search
Equilibrium.''} \emph{Journal of Political Economy} 90 (5): 881--94.

\bibitem[\citeproctext]{ref-doerr2023privacy}
Doerr, Sebastian, Leonardo Gambacorta, Luigi Guiso, and Marina Sanchez
del Villar. 2023. {``Privacy Regulation and Fintech Lending.''}
\emph{Available at SSRN 4353798}.

\bibitem[\citeproctext]{ref-farboodi2020long}
Farboodi, Maryam, and Laura Veldkamp. 2020. {``Long-Run Growth of
Financial Data Technology.''} \emph{American Economic Review} 110 (8):
2485--2523.

\bibitem[\citeproctext]{ref-Fernandez2022}
Fernandez, Marcelo Ariel, Kirill Rudov, and Leeat Yariv. 2022.
{``Centralized Matching with Incomplete Information.''} \emph{American
Economic Review: Insights} 4 (1): 18--33.
\url{https://doi.org/10.1257/aeri.20210123}.

\bibitem[\citeproctext]{ref-frankel2003equilibrium}
Frankel, David M, Stephen Morris, and Ady Pauzner. 2003. {``Equilibrium
Selection in Global Games with Strategic Complementarities.''}
\emph{Journal of Economic Theory} 108 (1): 1--44.

\bibitem[\citeproctext]{ref-galeotti2010network}
Galeotti, Andrea, Sanjeev Goyal, Matthew O Jackson, Fernando
Vega-Redondo, and Leeat Yariv. 2010. {``Network Games.''} \emph{The
Review of Economic Studies} 77 (1): 218--44.

\bibitem[\citeproctext]{ref-hellwig2009knowing}
Hellwig, Christian, and Laura Veldkamp. 2009. {``Knowing What Others
Know: Coordination Motives in Information Acquisition.''} \emph{The
Review of Economic Studies} 76 (1): 223--51.

\bibitem[\citeproctext]{ref-korinek2018artificial}
Korinek, Anton, and Joseph E Stiglitz. 2018. {``Artificial Intelligence
and Its Implications for Income Distribution and Unemployment.''} In
\emph{The Economics of Artificial Intelligence: An Agenda}, 349--90.
University of Chicago Press.

\bibitem[\citeproctext]{ref-mensch2020existence}
Mensch, Jeffrey. 2020. {``On the Existence of Monotone Pure-Strategy
Perfect Bayesian Equilibrium in Games with Complementarities.''}
\emph{Journal of Economic Theory} 187: 105026.

\bibitem[\citeproctext]{ref-milgrom1990rationalizability}
Milgrom, Paul, and John Roberts. 1990. {``Rationalizability, Learning,
and Equilibrium in Games with Strategic Complementarities.''}
\emph{Econometrica: Journal of the Econometric Society}, 1255--77.

\bibitem[\citeproctext]{ref-morris2003global}
Morris, Stephen, and Hyun Song Shin. 2003. {``Global Games: Theory and
Applications.''} In \emph{Advances in Economics and Econometrics: Theory
and Applications, Eighth World Congress, Volume 1}, 56--114. Cambridge
University Press.

\bibitem[\citeproctext]{ref-morris2016common}
Morris, Stephen, Hyun Song Shin, and Muhamet Yildiz. 2016. {``Common
Belief Foundations of Global Games.''} \emph{Journal of Economic Theory}
163: 826--48.

\bibitem[\citeproctext]{ref-myatt2012endogenous}
Myatt, David P, and Chris Wallace. 2012. {``Endogenous Information
Acquisition in Coordination Games.''} \emph{The Review of Economic
Studies} 79 (1): 340--74.

\bibitem[\citeproctext]{ref-quiggin2006supermodularity}
Quiggin, John, and Robert G Chambers. 2006. {``Supermodularity and Risk
Aversion.''} \emph{Mathematical Social Sciences} 52 (1): 1--14.

\bibitem[\citeproctext]{ref-roth2007repugnance}
Roth, Alvin E. 2007. {``Repugnance as a Constraint on Markets.''}
\emph{Journal of Economic Perspectives} 21 (3): 37--58.

\bibitem[\citeproctext]{ref-szkup2015information}
Szkup, Michal, and Isabel Trevino. 2015. {``Information Acquisition in
Global Games of Regime Change.''} \emph{Journal of Economic Theory} 160:
387--428.

\bibitem[\citeproctext]{ref-szkup2020sentiments}
---------. 2020. {``Sentiments, Strategic Uncertainty, and Information
Structures in Coordination Games.''} \emph{Games and Economic Behavior}
124: 534--53.

\bibitem[\citeproctext]{ref-szkup_trevino_2024}
---------. 2024. {``Selection Through Information Acquisition in
Coordination Games.''}

\bibitem[\citeproctext]{ref-tarski1955lattice}
Tarski, Alfred. 1955. {``A Lattice-Theoretical Fixpoint Theorem and Its
Applications.''} \emph{Pacific Journal of Mathematics} 5: 285--309.

\bibitem[\citeproctext]{ref-topkis1979equilibrium}
Topkis, Donald M. 1979. {``Equilibrium Points in Nonzero-Sum n-Person
Submodular Games.''} \emph{Siam Journal on Control and Optimization} 17
(6): 773--87.

\bibitem[\citeproctext]{ref-topkis1998supermodularity}
---------. 1998. \emph{Supermodularity and Complementarity}. Princeton
university press.

\bibitem[\citeproctext]{ref-van2007monotone}
Van Zandt, Timothy, and Xavier Vives. 2007. {``Monotone Equilibria in
Bayesian Games of Strategic Complementarities.''} \emph{Journal of
Economic Theory} 134 (1): 339--60.

\bibitem[\citeproctext]{ref-vives2005complementarities}
Vives, Xavier. 2005. {``Complementarities and Games: New
Developments.''} \emph{Journal of Economic Literature} 43 (2): 437--79.

\bibitem[\citeproctext]{ref-vives2007supermodularity}
---------. 2007. {``Supermodularity and Supermodular Games.''}
\emph{Occasional Paper}, no. 07/18.

\bibitem[\citeproctext]{ref-vives2024strategic}
Vives, Xavier, and Orestis Vravosinos. 2024. {``Strategic
Complementarity in Games.''} \emph{Journal of Mathematical Economics},
103005.

\bibitem[\citeproctext]{ref-yang2015coordination}
Yang, Ming. 2015. {``Coordination with Flexible Information
Acquisition.''} \emph{Journal of Economic Theory} 158: 721--38.

\end{CSLReferences}

\newpage

\section*{Appendix: basic definitions on supermodular
games}\label{appendix-basic-definitions-on-supermodular-games}
\addcontentsline{toc}{section}{Appendix: basic definitions on
supermodular games}

This appendix contains mathematical definitions that are helpful to
support the equilibrium analyses, but that might not be of interest to
all readers. It starts by defining lattice-theoretical objects and then
describes the supermodular games. It then restates a useful fixed point
theorem on lattices that is the basis of equilibria search in many
applications.

A key result underlying the usefulness of supermodular games is due to
Milgrom and Roberts (1990), who show (in their Theorem 5) that the set
of strategy profiles surviving iterated deletion of strongly dominated
strategies has a maximum and a minimum element, and both are Nash
equilibria.

\begin{definition}[Lattice]\protect\hypertarget{def-lattice}{}\label{def-lattice}

\ldots{}

\end{definition}

\begin{theorem}[Tarski (1955) fix
point]\protect\hypertarget{thm-fixedpoint}{}\label{thm-fixedpoint}

\ldots{}

\end{theorem}

\begin{definition}[Supermodular games (Vives
(2007))]\protect\hypertarget{def-supermodular}{}\label{def-supermodular}

Let players \(i= 1, \dots, N\) each have an action set \(A_i \in A\),
with the actual action taken by \(i\) denoted as \(a_i \in A\) and by
all the other players, \(a_{-i}\). \(A_i\) have the property of
component-wise order. The payoffs of each player are a continuous
mapping \(\pi_i : A^N \to \mathbb{R}\). A game \(G = (i, A_i, p_i)\) is
supermodular if:

\begin{enumerate}
\def\labelenumi{\arabic{enumi}.}
\item
  \(\pi_i\) is supermodular in \(a_i\) for fixed \(a_{-i}\), and
\item
  \(\pi_i\) displays increasing differences in \((a_i, a_{-i})\).
\end{enumerate}

\emph{Remark 1.} Due to the strategic complementarities, the best
responses in a supermodular game are monotone increasing, regardless of
whether \(\pi_i\) is quasiconcave in \(a_i\).

\end{definition}

\begin{definition}[Monotone supermodular games (Van Zandt and Vives
(2007))]\protect\hypertarget{def-monsupermodular}{}\label{def-monsupermodular}

In addition to the criteria in Definition~\ref{def-supermodular},
monotone supermodular games assume that each player's interim beliefs
\(p_i : T_i \to M_{-i}\) representing \(i\)'s set of probability
measures of the other players' types are increasing in first-order
stochastic dominance.

\emph{Remark 1.} If these criteria are met, the game has a greatest and
a least pure strategy Bayesian Nash equilibria.

\emph{Remark 2.} A first-order shift in the interim beliefs leads to
increases in these greatest and least equilibria.

\end{definition}

\begin{definition}[Bayesian Nash equilibrium
(BNE)]\protect\hypertarget{def-BNE}{}\label{def-BNE}

A BNE is a strategy profile such that each player \(i\) and their type
\(\theta_i\) chooses a best response to the strategy profile of other
players.

\end{definition}



\end{document}
